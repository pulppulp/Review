%%%%%%%%%%%%%%%%%%%%%%%%%%%%%%%%%%%%%%%%%
% Beamer Presentation
% LaTeX Template
% Version 1.0 (10/11/12)
%
% This template has been downloaded from:
% http://www.LaTeXTemplates.com
%
% License:
% CC BY-NC-SA 3.0 (http://creativecommons.org/licenses/by-nc-sa/3.0/)
%
%%%%%%%%%%%%%%%%%%%%%%%%%%%%%%%%%%%%%%%%%

%----------------------------------------------------------------------------------------
%	PACKAGES AND THEMES
%----------------------------------------------------------------------------------------

\documentclass{beamer}

\mode<presentation> {

% The Beamer class comes with a number of default slide themes
% which change the colors and layouts of slides. Below this is a list
% of all the themes, uncomment each in turn to see what they look like.

%\usetheme{default}
%\usetheme{AnnArbor}
%\usetheme{Antibes}
%\usetheme{Bergen}
%\usetheme{Berkeley}
%\usetheme{Berlin}
%\usetheme{Boadilla}
%\usetheme{CambridgeUS}
%\usetheme{Copenhagen}
%\usetheme{Darmstadt}
%\usetheme{Dresden}
%\usetheme{Frankfurt}
%\usetheme{Goettingen}
%\usetheme{Hannover}
%\usetheme{Ilmenau}
%\usetheme{JuanLesPins}
%\usetheme{Luebeck}
\usetheme{Madrid}
%\usetheme{Malmoe}
%\usetheme{Marburg}
%\usetheme{Montpellier}
%\usetheme{PaloAlto}
%\usetheme{Pittsburgh}
%\usetheme{Rochester}
%\usetheme{Singapore}
%\usetheme{Szeged}
%\usetheme{Warsaw}

% As well as themes, the Beamer class has a number of color themes
% for any slide theme. Uncomment each of these in turn to see how it
% changes the colors of your current slide theme.

%\usecolortheme{albatross}
%\usecolortheme{beaver}
%\usecolortheme{beetle}
%\usecolortheme{crane}
%\usecolortheme{dolphin}
%\usecolortheme{dove}
%\usecolortheme{fly}
%\usecolortheme{lily}
%\usecolortheme{orchid}
%\usecolortheme{rose}
%\usecolortheme{seagull}
%\usecolortheme{seahorse}
%\usecolortheme{whale}
%\usecolortheme{wolverine}

%\setbeamertemplate{footline} % To remove the footer line in all slides uncomment this line
%\setbeamertemplate{footline}[page number] % To replace the footer line in all slides with a simple slide count uncomment this line

%\setbeamertemplate{navigation symbols}{} % To remove the navigation symbols from the bottom of all slides uncomment this line
}

\usepackage{graphicx} % Allows including images
\usepackage{booktabs} % Allows the use of \toprule, \midrule and \bottomrule in tables

%----------------------------------------------------------------------------------------
%	TITLE PAGE
%----------------------------------------------------------------------------------------

\title[LNG 230/SPV 221 Language Acquisition]{Review Session} % The short title appears at the bottom of every slide, the full title is only on the title page

\author{Xiaomeng Ma} % Your name
\institute[Graduate Center, CUNY] % Your institution as it will appear on the bottom of every slide, may be shorthand to save space
{Graduate Center, CUNY \\ % Your institution for the title page
\medskip
\textit{xma3@gradcenter.cuny.edu} % Your email address
}
\date{October 3, 2018} % Date, can be changed to a custom date

\begin{document}

\begin{frame}
\titlepage % Print the title page as the first slide
\end{frame}


%----------------------------------------------------------------------------------------
%	PRESENTATION SLIDES
%----------------------------------------------------------------------------------------

%------------------------------------------------
\section{First Section} % Sections can be created in order to organize your presentation into discrete blocks, all sections and subsections are automatically printed in the table of contents as an overview of the talk
%------------------------------------------------

\subsection{Subsection Example} % A subsection can be created just before a set of slides with a common theme to further break down your presentation into chunks

\begin{frame}{Introduction}
\begin{itemize}
\item What exactly does baby need to do to acquire language?
\pause 
    \begin{itemize}
        \item Comprehension: Speech Recognition, Word meaning mapping...
        \item Production: Speech production, Grammar...
\end{itemize}
\pause
\item What is language
\begin{itemize}
    \item Semanticity:
    \pause Symbols have (fixed) meanings
    \pause 
    \item Arbitariness: 
    \pause Symbols are randomly assigned to meanings.
    \pause 
    \item Productivity:
    \pause Symbols should be able to create infinite meanings.
    \pause 
    \item Displacement:
    \pause Symbols can be used to describe events that are not present.
\end{itemize}
\end{itemize}
\end{frame}
%------------------------------------------------
\begin{frame}
\frametitle{Research Methods in Language Acquisition}
\begin{itemize}
\item  What kind of data do we need to analyze child language acquisition
\pause 
    \begin{itemize}
        \item Production
        \pause
        \item Comprehension
    \end{itemize}
\pause
\item Research Methods:
\pause
\begin{itemize}
        \item Behavioral Methods
        \pause
        \item Eye-tracking
        \pause 
        \item ERP
        \pause
        \item Corpus Analysis
    \end{itemize}
\end{itemize}
\end{frame}
%------------------------------------------------
\begin{frame}
\frametitle{Research Methods in Language Acquisition}
\begin{itemize}
\item Behavioral Methods
\pause
\begin{itemize}
    \item Production:
    \pause
    \begin{itemize}
        \item Naturalistic Studies
        \item Production Experiment
    \end{itemize}
    \pause
    \item Comprehension:
    \pause
    \begin{itemize}
        \item Truth Value Judgment
        \item Picture Matching Task
        \item Act-Out Task
    \end{itemize}
\end{itemize}
\end{itemize}
\end{frame}
%------------------------------------------------
\begin{frame}
\frametitle{Research Methods in Language Acquisition}
\begin{itemize}
\item Eyetracking: fine-grained temporal analysis of eye gaze grounded with respect to specific language stimuli
\pause 
\begin{itemize}
    \item People tend to direct their eye gaze to things they are attending to in their visual environment
    \item Saccade: Rapid eye-movement
    \item Fixation: Visual gaze on a single location.
\end{itemize}
\end{itemize}
\end{frame}
%------------------------------------------------
\begin{frame}
\frametitle{Research Methods in Language Acquisition}
\begin{itemize}
\item ERP: Event-related Potentials  provide a non-invasive way of examining the neurophysiological correlates of language processing
\pause
\begin{itemize}
    \item Auditory processing:
    \pause
    Mismatched Negativity (MMN)
    \pause
    \item Semantic processing:
    \pause 
    Negative deflection around 400ms (N400)
    \pause
    \item Syntax processing:
    \pause
    Positive deflection around 600ms (P600)
\end{itemize}
\end{itemize}
\end{frame}
%------------------------------------------------
\begin{frame}
\frametitle{Research Methods in Language Acquisition}
\begin{itemize}
\item MLU:Mean Length of Utterances
\item Rules:
\begin{itemize}
    \item Do not count such fillers as \textit{mm} or \textit{oh}, but do count \textit{no}, \textit{yeah}, and \textit{hi}.
    \item All compound words (two or more free morphemes), proper names, and ritualized reduplications count as single words.
    \item Count as one morpheme all irregular pasts of the verb (got, did, went, saw).
    \item Count as one morpheme all diminutives (doggie, mommie) 
    \item Count as separate morphemes all auxiliaries (is, have, will, can, must, would). Also all catenatives: gonna, wanna, hafta. 
    \item Count as separate morphemes all inflections, for example, possessive {s}, plural {s}, third person singular {s}, regular past {d}, progressive {ing}.
\end{itemize}
\end{itemize}
\end{frame}
%------------------------------------------------
\begin{frame}
\frametitle{MLU Calculation Practice}
58	*CHI:	hi .
\pasue (1)\\
59	*CHI:	drop it (.) you stupid pusher .
\pause (2,3)\\
60	*CHI:	I wonder how it got tangled up .
\pause (8)\\
\pause
Utterances: 4\\
Morphemes: 1+2+3+8 = 14\\
MLU: 14/4 = 3.5\\
\end{frame}
%------------------------------------------------
\begin{frame}
\frametitle{Speech}
\begin{itemize}
    \item Children usuall taks around 10-18 months
    \item What happened in the first 12 months for speech acquisition?
    \begin{itemize}
        \item identifying meaningful sounds
        \item mapping speech stream onto meaningful units
        \item matching speech sounds and trying to produce
    \end{itemize}
\end{itemize}
\end{frame}
%------------------------------------------------
\begin{frame}
\frametitle{Speech}
\begin{itemize}
    \item Identifying Meaningful Sounds
    \begin{itemize}
        \item Speech Pattern Recognition starts prenatal
        \item Phoneme: smallest units of sound that have contrastive meanings (pat vs bat)
        \item Difficulties in Learning Phonemes:
        \pause
        \begin{itemize}
            \item Lack of acoustic invariance
            \item Allopohnes
        \end{itemize}
    \end{itemize}
\end{itemize}
\end{frame}
%------------------------------------------------
\begin{frame}
\frametitle{Theories for Identifying Meaningful Sounds}
\begin{itemize}
\item \textbf{Motor Theory:} 
\pause Everthing's innate
\begin{itemize}
    \item Infants also have ability to identify musical tones
    \item Animals also have categorical perception
\end{itemize}
\pause
\item \textbf{Universal Theory:}
\pause We are born with innate system,  but the innate ability comes from all-purpose perceptual mechanisms (not linguistically specific) 
\item Evidence: Infants are tuned to their native languages
\pause 
\item Problem:
\begin{itemize}
    \item We do not lose the ability to distinguish non-native contrasts completely.
    \item That infants are not born with the ability to distinguish all the contrasts of all languages.
\end{itemize}
\item \textbf{Attuenment Theory}:
\pause
We are born with sensitivity to certain differences but that others have to be created or fine-tuned from our exposure to our native language.
\end{itemize}
\end{frame}
%------------------------------------------------
\begin{frame}{Speech Segmentation}
\begin{itemize}
    \item Children use cues to segment speech.
    \pause
    \begin{itemize}
        \item Prosodic cues:
        \pause
        7.5 month old infants can identify trochaic pattern in English words
        \item Phonotactic Regularities:
        \pause /vzg/ rare in English, common in Russian
        \item Allophonic Variation:
        \pause
        10 month old infants can discriminate allophones
    \end{itemize}
    \pause
    \item Problems:
    \begin{itemize}
        \item Mis-segment (atomic vs a tomic)
        \item Couldn't explain crosslinguistic differences
    \end{itemize}
\end{itemize}
\end{frame}
%------------------------------------------------
\begin{frame}{Speech Segmentation}
\begin{itemize}
    \item Children could also use 
    \textbf{isolated words} to learn how to segment speech.
    \item Children could also apply \textbf{Transitional Probability} to identify words in speech
\end{itemize}
\end{frame}
%------------------------------------------------
\begin{frame}{Speech Production}
\begin{itemize}
    \item We make sounds through vocal tract
    \pause
    \item Infants have different vocal tract than adults.
    \item Features in Infants Vocalization:
    \begin{itemize}
        \item Sounds are omitted (e.g broke vs bok)
        \item Sounds are substituted for each other (e.g rabbit vs wabbit)
        \item Consonant harmony (e.g doggy vs goggy)
        \item Repeat syllables within words (e.g. tummy vs tum tum ; bottle vs baba)
    \end{itemize}
\end{itemize}
\end{frame}
%------------------------------------------------
\begin{frame}{Speech Production Theory}
\begin{itemize}
    \item \textbf{Mispronunciation due to misperception:}
    \pause
    Children could not differentiate similar phonemes.
    \pause
    \item Problem: 
    \pause
     Children are sensitive to not only phonemes, but also allophonic variation at a young age.
     \item \textbf{Articulatory constraints on production}
     \pause 
     Infants can't make certain voices because they have immature vocal tract
     \item Problem:
     \pause
     Puddle-Puggle \\
     Puzzle-Puddle
\end{itemize}
\end{frame}
%------------------------------------------------
\begin{frame}{Speech Production Theory}
\begin{itemize}
    \item \textbf{Universal Constraints:}
    \pause
    Development of Speech sounds is governed by a genetically programmed universal sequence of maturation.
    \item Marked Sounds are acquired early, unmarked sounds are acquired late
    \item Speech sound should be acquired in the same order in different languages
    \item \textbf{Template Theory:}
    \begin{itemize}
        \item Language acquisition is task-oriented
        \item However, the task is too difficult since there are too much to learn and children unable to produce speech accurately 
        \item They use well-practiced word patterns to replace the utterences they are unable to produce accurately
    \end{itemize}
\end{itemize}
\end{frame}
%------------------------------------------------
\begin{frame}{Word Acquisition}
\begin{itemize}
    \item Children have amazing ability to learn words
    \item Problems in Word Acquisition:
    \begin{itemize}
        \item Reference: (The Gavagai problem)
        \pause \\ 
        With each referential act, there are an infinite number of hypotheses for the meaning of the word dog that are consistent with the data
        \pause
        \item Extension:
        \pause 
        Words are not always denote tangible things. They are used to denote categories/concepts.
    \end{itemize}
\end{itemize}
\end{frame}
%------------------------------------------------
\begin{frame}{Theories in Word Acquistion}
\begin{itemize}
    \item \textbf{Constraints Theory}:
    \pause
    childrens word learning is guided by a set of default assumptions or constraints on hypotheses 
    \pause
    \begin{itemize}
        \item Whole item assumption: 
        \pause 
        n the first instance, learners should assume that new words refer to whole objects, rather than parts of objects, actions, events or spatial relations.
        \pause
        \item The mutual exclusivity assumption:
        \pause
        Learners should assume that objects only have one name.
    \end{itemize}
    \item Developmental Lexical Framework
\end{itemize}
\end{frame}
%------------------------------------------------
\begin{frame}{Theories in Word Acquisition}
\begin{itemize}
    \item \textbf{Social Pragmatic Account:}
    \pause
    Children do not need innate linguistic constraints. Children pay attention to ”social clues” to learn the meaning of new words.
    \begin{itemize}
        \item joint attention
        \item communicative intent
    \end{itemize}
    \item \textbf{Attentional Learning Account:}
    \pause
    Children can rely on associateive learning ability to learn words
    \item \textbf{Syntactic Bootstrapping} (verb acquisition)
    \pause
    that children exploit (possibly universal) mappings between meaning (semantics) and syntax to learn verbs through constraints of sentence types
    \item \textbf{Emergentist Coalition Model (ECM)}:
    \pause a hybrid account that is sensitive to the multiple strategies children use to break the word barrier and to move from being novice to expert learners
\end{itemize}
    
\end{frame}

\end{document}